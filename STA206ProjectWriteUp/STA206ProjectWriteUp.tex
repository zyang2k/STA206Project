% Options for packages loaded elsewhere
\PassOptionsToPackage{unicode}{hyperref}
\PassOptionsToPackage{hyphens}{url}
% !TeX program = pdfLaTeX
\documentclass[12pt]{article}
\usepackage{amsmath}
\usepackage{graphicx,psfrag,epsf}
\usepackage{enumerate}
\usepackage[]{natbib}
\usepackage{textcomp}


%\pdfminorversion=4
% NOTE: To produce blinded version, replace "0" with "1" below.
\newcommand{\blind}{0}

% DON'T change margins - should be 1 inch all around.
\addtolength{\oddsidemargin}{-.5in}%
\addtolength{\evensidemargin}{-1in}%
\addtolength{\textwidth}{1in}%
\addtolength{\textheight}{1.7in}%
\addtolength{\topmargin}{-1in}%

%% load any required packages here



% tightlist command for lists without linebreak
\providecommand{\tightlist}{%
  \setlength{\itemsep}{0pt}\setlength{\parskip}{0pt}}




\IfFileExists{bookmark.sty}{\usepackage{bookmark}}{\usepackage{hyperref}}
\IfFileExists{xurl.sty}{\usepackage{xurl}}{} % add URL line breaks if available
\hypersetup{
  pdftitle={Exploring Determinants of Plasma Retinol and Beta-Carotene Levels: The Role of Personal Characteristics and Dietary Intake},
  hidelinks,
  pdfcreator={LaTeX via pandoc}}



\begin{document}


\def\spacingset#1{\renewcommand{\baselinestretch}%
{#1}\small\normalsize} \spacingset{1}


%%%%%%%%%%%%%%%%%%%%%%%%%%%%%%%%%%%%%%%%%%%%%%%%%%%%%%%%%%%%%%%%%%%%%%%%%%%%%%

\if0\blind
{
  \title{\bf Exploring Determinants of Plasma Retinol and Beta-Carotene
Levels: The Role of Personal Characteristics and Dietary Intake}

  \author{
        Ziyue Yang \\
    \\
     and \\     Xinyi Cen \\
    \\
      }
  \maketitle
} \fi

\if1\blind
{
  \bigskip
  \bigskip
  \bigskip
  \begin{center}
    {\LARGE\bf Exploring Determinants of Plasma Retinol and
Beta-Carotene Levels: The Role of Personal Characteristics and Dietary
Intake}
  \end{center}
  \medskip
} \fi

\bigskip
\begin{abstract}
This is a test abstract.
\end{abstract}

\noindent%
 

\vfill

\newpage
\spacingset{1.9} % DON'T change the spacing!

\section{Introduction}\label{introduction}

Vitamin A is an essential nutrient that supports critical physiological
functions, including vision, immune response, reproduction, and cellular
communication\citep{Blaner2020, Ross2014, InstituteofMedicine2000}.
Retinol is the major biomarker used to assess vitamin A levels in
humans\citep{Blaner2020}. It plays an essential role in maintaining
healthy skin, supporting immune function, and regulating gene
expression\citep{huang_role_2018}. Beta-carotene can be enzymatically
converted to retinol in the body, making it an important dietary source
of vitamin A\citep{grune_carotene_2010}. In addition, beta-carotene also
acts as a powerful antioxidant\citep{fiedor_potential_2014}.

This study analyzes data from a cross-sectional study designed by
Therese Stukel, which includes 315 patients who underwent elective
surgical procedures over a three-year period. These procedures involved
biopsies or the removal of non-cancerous lesions from organs such as the
lung, colon, breast, skin, ovary, or
uterus\citep{nierenberg_determinants_1989}.

Given the known associations between plasma levels of vitamin A and the
risk of various cancers, understanding the determinants of plasma
retinol and beta-carotene levels is critical. Investigating factors such
as personal characteristics (e.g., sex, age, smoking status, BMI) and
dietary intake (e.g., retinol, beta-carotene, and calorie consumption)
will provide invaluable insights into the biological and lifestyle
influences on these important biomarkers.

\section{Methods and Results}\label{methods-and-results}

We first created various visualizations as part of our exploratory data
analysis to help understand who is in the data and acknowledge potential
sources of bias within the data.

\subsection{\texorpdfstring{Outcome Variable of Interest
(\texttt{retplasma},
\texttt{betaplasma})}{Outcome Variable of Interest (retplasma, betaplasma)}}\label{outcome-variable-of-interest-retplasma-betaplasma}

The variables \texttt{retplasma} and \texttt{betaplasma} represent
plasma beta-carotene and retinol levels in ng/mL respectively. According
to the summary statistics, both variables exhibit large ranges,
suggesting heterogeneity in the population. Skewness is evident in
histograms of these two variables, particularly for \texttt{betaplasma}.
The wide range and skewness suggest the need for a log transformation to
normalize distributions and stabilize variance.

\subsection{\texorpdfstring{Personal Characteristic Variables
(\texttt{sex}, \texttt{age}, \texttt{smoke},
\texttt{bmi})}{Personal Characteristic Variables (sex, age, smoke, bmi)}}\label{personal-characteristic-variables-sex-age-smoke-bmi}

The dataset includes several personal characteristics that may influence
plasma retinol and beta-carotene levels.

\begin{itemize}
\tightlist
\item
  \textbf{Age}, a numerical variable, ranges from 19 to 83 years, with a
  median of 48 years and a mean of 50.18 years, indicating a slightly
  right-skewed distribution where older individuals are slightly
  overrepresented.
\item
  \textbf{Sex} is a categorical variable, with the majority of
  participants being female (272 out of 314 individuals), highlighting a
  potential gender imbalance in the dataset.
\item
  \textbf{Smoking status (smoke)} is another categorical variable, with
  most individuals categorized as nonsmokers. Among the participants,
  about half have never smoked, followed by a significant proportion of
  former smokers, while current smokers are the least represented group.
\item
  \textbf{BMI (body mass index)}, a numerical variable, ranges from
  16.33 to 50.40 kg/\(m^2\), with a median of 24.71 kg/\(m^2\) and a
  mean of kg/\(m^2\), suggesting a slight skew toward higher BMI values.
  Based on BMI categories, individuals are classified as underweight
  (BMI \textless{} 18.5), normal weight (18.5 \(\leq\) BMI \textless{}
  25), overweight (25 \(\leq\) BMI \textless{} 30), or obese (BMI
  \(\geq\) 30). The majority of individuals fall within the normal
  weight and overweight categories, with a smaller proportion classified
  as obese or underweight.
\end{itemize}

These variables provide a snapshot of the demographic and health-related
characteristics of the study population, which may play a role in
influencing plasma biomarker levels.

\subsection{Dietary Intake Variables}\label{dietary-intake-variables}

The dataset includes several dietary intake variables that may influence
plasma retinol and beta-carotene levels.

\begin{itemize}
\tightlist
\item
  \textbf{Vituse} (vitamin use) is a categorical variable, with most
  participants reporting some level of vitamin use. \textbf{Calories}
  range from 445.2 to 6662.2 kcal/day, with a mean of 1792.8 kcal/day
  and a median of 1665 kcal/day, indicating moderate variability in
  daily energy intake.
\item
  \textbf{Fat} intake ranges from 14.4 to 235.9 grams/day, with a mean
  of 76.75 grams/day and a median of 72.9 grams/day.
\item
  \textbf{Fiber} intake ranges from 3.1 to 36.8 grams/day, with a median
  of 12.1 grams/day, suggesting that many participants fall below
  recommended daily fiber intake levels.
\item
  \textbf{Alcohol} consumption shows a heavily right-skewed
  distribution, ranging from 0 to 203 drinks/week, with a mean of 3.29
  drinks/week and a median of 0.3 drinks/week, indicating that most
  participants consume little or no alcohol.
\item
  \textbf{Cholesterol} intake ranges from 37.7 to 814.7 mg/day, with a
  mean of 240.4 mg/day and a median of 206.2 mg/day, reflecting
  variability in dietary patterns.
\item
  \textbf{Betadiet}, representing dietary beta-carotene intake, ranges
  from 214 to 9642 \(\mu\)g/day, with a mean of 2189 \(\mu\)g/day,
  indicating substantial variation among participants. Similarly,
\item
  \textbf{Retdiet}, dietary retinol intake, ranges from 30 to 6901
  \(\mu\)g/day, with a mean of 825.6 \(\mu\)/day and a median of 707
  \(\mu\)g/day.
\end{itemize}

These dietary variables exhibit a wide range of values, reflecting
diverse dietary habits within the study population, and are critical for
understanding their impact on plasma biomarker levels.

\bibliographystyle{apalike}
\bibliography{bibliography.bib}



\end{document}
